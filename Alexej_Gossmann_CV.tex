%%%%%%%%%%%%%%%%%%%%%%%%%%%%%%%%%%%%%%%%%
% Medium Length Graduate Curriculum Vitae
% LaTeX Template
% Version 1.1 (9/12/12)
%
% This template has been downloaded from:
% http://www.LaTeXTemplates.com
%
% Original author:
% Rensselaer Polytechnic Institute (http://www.rpi.edu/dept/arc/training/latex/resumes/)
%
% Important note:
% This template requires the res.cls file to be in the same directory as the
% .tex file. The res.cls file provides the resume style used for structuring the
% document.
%
%%%%%%%%%%%%%%%%%%%%%%%%%%%%%%%%%%%%%%%%%

%----------------------------------------------------------------------------------------
% PACKAGES AND OTHER DOCUMENT CONFIGURATIONS
%----------------------------------------------------------------------------------------

\documentclass[overlapped, line, 10pt]{res} % Use the res.cls style, the font size can be changed to 11pt or 12pt here

\usepackage{helvet} % Default font is the helvetica postscript font
\usepackage{hyperref}
%\usepackage{newcent} % To change the default font to the new century schoolbook postscript font uncomment this line and comment the one above

\setlength{\textwidth}{5.5in} % Text width of the document

% Makes the random asterisk (*), that appears above the bibtex references for some reason, disappear.
% See http://tex.stackexchange.com/questions/28339/random-asterisk-when-using-bibtex-and-res-cls
\makeatletter
\renewenvironment{thebibliography}[1]
     {\list{\@biblabel{\@arabic\c@enumiv}}%
           {\settowidth\labelwidth{\@biblabel{#1}}%
            \leftmargin\labelwidth
            \advance\leftmargin\labelsep
            \@openbib@code
            \usecounter{enumiv}%
            \let\p@enumiv\@empty
            \renewcommand\theenumiv{\@arabic\c@enumiv}}%
      \sloppy
      \clubpenalty4000
      \@clubpenalty \clubpenalty
      \widowpenalty4000%
      \sfcode`\.\@m}
     {\def\@noitemerr
       {\@latex@warning{Empty `thebibliography' environment}}%
      \endlist}
\makeatother

\begin{document}

%----------------------------------------------------------------------------------------
% NAME AND ADDRESS SECTION
%----------------------------------------------------------------------------------------

\name{Alexej Gossmann}

\address{
  Web: \href{http://www.alexejgossmann.com/}{alexejgossmann.com} \textbar
  LinkedIn: \href{https://www.linkedin.com/in/alexejgossmann/}{alexejgossmann} \textbar
  Github: \href{https://github.com/agisga}{agisga} \textbar
  Email: \href{mailto:agossman@tulane.edu}{agossman@tulane.edu}
}

%----------------------------------------------------------------------------------------

\begin{resume}

%---

%\section{OBJECTIVE}

%---

\section{AREAS OF INTEREST}

Development of statistical and machine learning techniques for feature selection and prediction on big high-dimensional datasets, with focus on understanding how false positive findings can arise, and how they can be avoided;
applications in genomics and neuroimaging, and other medical applications of machine learning;
translation of the statistical and machine learning methodology into a useful (software) product.

%---

\section{EDUCATION}

{\sl PhD,} Bioinnovation \\
Tulane University, New Orleans, Louisiana, in progress

{\sl Doctoral research,} Mathematics \\
Tulane University, New Orleans, Louisiana, through January 2017\\
GPA: 3.978

{\sl MS,} Statistics \\
Tulane University, New Orleans, Louisiana, May 2014\\
GPA: 3.975

{\sl BS,} Mathematics\\
Technische Universit\"{a}t Darmstadt, Darmstadt, Germany, May 2012\\
%Thesis: On disjunction and numerical existence properties of extensions of Heyting arithmetic (supervised by Dr. Ulrich Kohlenbach)\\
GPA: 3.7 (1.54 German grade)

%---

\section{SKILLS}

{\sl Software, programming, tools:} R, Ruby, C++, Python, Matlab, \LaTeX, Linux/Unix, git and github, HTML, CSS.\\
{\sl Domain knowledge:} Genomics and medical imaging.\\
{\sl Language knowledge:} Bilingual in German and Russian, fluent in English.

%---

\section{EXPERIENCE}

\title{Research assistantship (doctoral research)}
\employer{The Multiscale Bioimaging and Bioinformatics Laboratory}
\dates{Jan 2015 -- present}
\location{Tulane University, New Orleans, LA}
\begin{position}
  Research in statistics and machine learning with application in genomics and neuroimaging under the supervision of Dr. Yu-Ping Wang, resulting in six peer-reviewed publications (\cite{gossmann2015}, \cite{cao2015BCB}, \cite{cao2015bioinformatics}, \cite{Gossmann2017-yu}, \cite{Gossmann2017-ln}, \cite{brzyski2016}), presentations at multiple conferences and workshops, and several open source software packages.
\end{position}

\title{Student intern (ORISE)}
\employer{Division of Imaging, Diagnostics, and Software Reliability (CDRH/OSEL/DIDSR) at the U.S. Food \& Drug Administration}
\dates{May -- Aug 2017, and Jan -- Feb 2018}
\location{FDA, Silver Spring, MD}
\begin{position}
  Machine learning research resulting in a conference presentation and associated publication \cite{gossmann2018} that can inform future policy on evaluation of machine learning systems in medicine.
\end{position}

\title{Statistics MS research project}
\employer{Department of Mathematics}
\location{Tulane University, New Orleans, LA}
\dates{Sept 2013 -- May 2014}
\begin{position}
  Analysis of bone growth data using mixed-effects smoothing spline ANOVA methods (supervised by Dr. Michelle Lacey) resulting in a journal publication \cite{sammarco2015}.
\end{position}

\title{Research assistantship}
\employer{Department of Mathematics}
\location{Tulane University, New Orleans, LA}
\dates{Jun -- Aug 2014}
\begin{position}
  Worked under the supervision of Dr. Oleksandr Gromenko on statistical methods for the analysis of spatio-temporal processes with application in weather prediction.
\end{position}

\title{Student developer}
\employer{Google Summer of Code 2015}
\dates{May -- August 2015}
\location{Remote position}
\begin{position}
  Project: Adding Linear Mixed Effects Models Support to SciRuby (supervised by developers from the Ruby Science Foundation). The created statistical software package received about 2700 downloads, and was considered for deployment at \url{http://www.genenetwork.org}.
\end{position}

\title{Mentor}
\employer{Google Summer of Code 2016 (with the Ruby Science Foundation)}
\location{Remote position}
\dates{May -- August 2016}
\begin{position}
  Involved with project proposals, selection/interviewing of students, mentoring. Mentored project: Categorical data support for Daru, Statsample and Statsample-glm. The mentored student (Lokesh Sharma) made major improvements to several open source software packages for data analysis in Ruby.
\end{position}

\title{Mentor}
\employer{Google Summer of Code 2017 (with the Ruby Science Foundation)}
\location{Remote position}
\dates{May -- August 2017}
\begin{position}
  Involved with project proposals, selection/interviewing of students, mentoring. Mentored project: Creating the fastest math libraries for Ruby by using the GPU through OpenCL and ArrayFire. Outcome: a Ruby library for scientific computing on the GPU (\url{https://github.com/arrayfire/arrayfire-rb}) and multiple conference presentations by the mentored student (Prasun Anand).
\end{position}

\title{Instructor, co-teacher, teaching assistant}
\employer{Technische Universit\"{a}t Darmstadt, and Tulane University}
\location{Darmstadt, Germany, and New Orleans, LA}
\dates{Fall semester 2010 -- Fall semester 2014}
\begin{position}
  Instructor for Calculus 1, co-teacher for Statistics for Scientists, and teaching assistant for Real Analysis 1 and 2 as well as for various undergraduate courses in mathematics and statistics.
\end{position}

%---

\section{PUBLICATIONS}

\bibliographystyle{apalike}
\bibliography{Alexej_Gossmann_CV}

%---

\section{CONFERENCE PRESENTATIONS}

\begin{itemize} \itemsep -2pt % Reduce space between items
  \item SlopeCCA and gslopeCCA: sorted L1 penalized canonical correlation analysis (Abstract/Program \#2803W). Presented at the {\it 66th Annual Meeting of The American Society of Human Genetics}, October 19, 2016, Vancouver, Canada.
  \item Identification of Significant Genetic Variants via SLOPE, and its Extension to Group SLOPE (Abstract/Program \#1343F). Presented at the {\it 65th Annual Meeting of The American Society of Human Genetics}, October 9, 2015, Baltimore, MD.
  \item Identification of Significant Genetic Variants via SLOPE, and its Extension to Group SLOPE. The {\it 6th ACM Conference on Bioinformatics, Computational Biology, and Health Informatics}, Atlanta, GA, September 2015.
\end{itemize}

%---

\section{SOFTWARE}

\begin{itemize} \itemsep -2pt % Reduce space between items
    \item \verb!grpSLOPE! -- Group SLOPE (Group Sorted L1 Penalized Estimation) is a penalized linear regression method that is used for adaptive selection of groups of significant predictors in a high-dimensional linear model. Group SLOPE can control the (group) false discovery rate at a user-specified level. Project repository: \url{https://github.com/agisga/grpSLOPE}. CRAN: \url{https://cran.r-project.org/web/packages/grpSLOPE/index.html}.
    \item \verb!grpSLOPEMC! -- This is an extension package to the R package grpSLOPE. It contains Monte Carlo based methods for the estimation of the regularizing sequence. Project repository: \url{https://github.com/agisga/grpSLOPEMC}.
    \item \verb!mixed_models! -- Fit statistical linear models with fixed and random effects in Ruby. Project repository: \url{https://github.com/agisga/mixed_models}. RubyGems: \url{https://rubygems.org/gems/mixed_models}.
    \item In my free time I like to contribute to open source software projects. Visit my Github page for the projects that I contribute to: \url{https://github.com/agisga}.
\end{itemize}

%---

\section{EXTRA-CURRICULAR ACTIVITIES}

Elected {\it President} of the SIAM student chapter at Tulane University, September 2014 -- September 2016.
Organized the Graduate Student Colloquium at the Department of Mathematics, Tulane University, September 2014 -- September 2016.
Participation in \emph{Google Summer of Code} as a student developer in 2015, and as a mentor in 2016 and 2017.
Participation in the middle school outreach program \emph{Girls in STEM at Tulane (GIST)}, November 2017.

%---

\section{HONORS \& AWARDS}

\begin{itemize} \itemsep -2pt % Reduce space between items
  \item Reviewers’ Choice Abstract (top 10\% of poster abstracts), 66th Annual Meeting of The American Society of Human Genetics, October 19, 2016, Vancouver, Canada.
  \item SIAM Student Chapter Certificate of Recognition, SIAM (Society for Industrial and Applied Mathematics), April 2015.
\end{itemize}

\vfill
\moveleft.5\hoffset\centerline{Updated: \today}

%----------------------------------------------------------------------------------------

\end{resume}

\end{document}
