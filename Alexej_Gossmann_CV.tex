%%%%%%%%%%%%%%%%%%%%%%%%%%%%%%%%%%%%%%%%%
% Medium Length Graduate Curriculum Vitae
% LaTeX Template
% Version 1.1 (9/12/12)
%
% This template has been downloaded from:
% http://www.LaTeXTemplates.com
%
% Original author:
% Rensselaer Polytechnic Institute (http://www.rpi.edu/dept/arc/training/latex/resumes/)
%
% Important note:
% This template requires the res.cls file to be in the same directory as the
% .tex file. The res.cls file provides the resume style used for structuring the
% document.
%
%%%%%%%%%%%%%%%%%%%%%%%%%%%%%%%%%%%%%%%%%

%----------------------------------------------------------------------------------------
% PACKAGES AND OTHER DOCUMENT CONFIGURATIONS
%----------------------------------------------------------------------------------------

% \documentclass[margin, 10pt]{res} % Use the res.cls style, the font size can be changed to 11pt or 12pt here
\documentclass[margin]{res} % Use the res.cls style, the font size can be changed to 11pt or 12pt here

\usepackage{helvet} % Default font is the helvetica postscript font
\usepackage{hyperref}
%\usepackage{newcent} % To change the default font to the new century schoolbook postscript font uncomment this line and comment the one above

\setlength{\textwidth}{5.1in} % Text width of the document

% Makes the random asterisk (*), that appears above the bibtex references for some reason, disappear.
% See http://tex.stackexchange.com/questions/28339/random-asterisk-when-using-bibtex-and-res-cls
\makeatletter
\renewenvironment{thebibliography}[1]
     {\list{\@biblabel{\@arabic\c@enumiv}}%
           {\settowidth\labelwidth{\@biblabel{#1}}%
            \leftmargin\labelwidth
            \advance\leftmargin\labelsep
            \@openbib@code
            \usecounter{enumiv}%
            \let\p@enumiv\@empty
            \renewcommand\theenumiv{\@arabic\c@enumiv}}%
      \sloppy
      \clubpenalty4000
      \@clubpenalty \clubpenalty
      \widowpenalty4000%
      \sfcode`\.\@m}
     {\def\@noitemerr
       {\@latex@warning{Empty `thebibliography' environment}}%
      \endlist}
\makeatother
\begin{document}

%----------------------------------------------------------------------------------------
% NAME AND ADDRESS SECTION
%----------------------------------------------------------------------------------------

\moveleft.5\hoffset\centerline{\large\bf Alexej Gossmann} % Your name at the top

\moveleft.5\hoffset\centerline{Bioinnovation PhD Program, Tulane University, New Orleans} % Your address

\moveleft\hoffset\vbox{\hrule width\resumewidth height 1pt}\smallskip % Horizontal line after name; adjust line thickness by changing the '1pt'


\vspace{5pt}
\moveleft.5\hoffset\centerline{{Web:} \href{http://www.alexejgossmann.com/}{alexejgossmann.com} \textbar {LinkedIn:} \href{https://www.linkedin.com/in/alexejgossmann/}{alexejgossmann} \textbar {Github:} \href{https://github.com/agisga}{agisga} \textbar {Email:} \href{mailto:agossman@tulane.edu}{agossman@tulane.edu}}

%----------------------------------------------------------------------------------------

\begin{resume}

%---

%\section{OBJECTIVE}

%---

\section{AREAS OF INTEREST}

Development of statistical and machine learning techniques for feature selection and prediction on big high-dimensional datasets, with focus on understanding how false positive findings can arise, and how they can be avoided;
applications in genomics and neuroimaging, and other medical applications of machine learning;
translation of the statistical and machine learning methodology into a useful (software) product.

%---

\section{EDUCATION}

{\sl PhD,} Bioinnovation \\
Tulane University, New Orleans, Louisiana, in progress

{\sl Doctoral research,} Mathematics \\
Tulane University, New Orleans, Louisiana, through January 2017\\
GPA: 3.978

{\sl MS,} Statistics \\
Tulane University, New Orleans, Louisiana, May 2014\\
%Master's Research Project: Analysis of Bone Growth Data with Mixed-Effects Smoothing Spline ANOVA Methods (supervised by Dr. Michelle Lacey)\\
GPA: 3.975

{\sl BS,} Mathematics\\
Technische Universit\"{a}t Darmstadt, Darmstadt, Germany, May 2012\\
%Thesis: On disjunction and numerical existence properties of extensions of Heyting arithmetic (supervised by Dr. Ulrich Kohlenbach)\\
GPA: 3.7 (1.54 German grade)

%---

\section{SKILLS}

{\sl Software, programming, tools:} R, Ruby, C++, Python, Matlab, \LaTeX, Linux/Unix, git and github, HTML, CSS.\\
{\sl Domain knowledge:} Genomics and medical imaging.\\
{\sl Language knowledge:} Bilingual in German and Russian, fluent in English.

%---

\section{EXPERIENCE}

\begin{itemize}
  \item Research assistantship in The Multiscale Bioimaging and Bioinformatics Laboratory at Tulane University under the supervision of {Dr.\,Yu-Ping Wang}. Spring 2015 -- present.
  \item Student Intern doing machine learning research in the Division of Imaging, Diagnostics, and Software Reliability (CDRH/OSEL/DIDSR) at the U.S. Food \& Drug Administration (FDA). May -- August 2017 and Jan -- Feb 2018.
  \item Instructor, co-teacher, or teaching assistant for various undergraduate statistics, calculus, and real analysis courses at Tulane University and Technische Universit\"{a}t Darmstadt, Fall 2010 - Fall 2014.
  \item Student developer for Google Summer of Code 2015. Project: Adding Linear Mixed Effects Models Support to SciRuby. Supervised by the Ruby Science Foundation. May -- August 2015.
  \item Mentor for Google Summer of Code 2016. Project: Categorical data support for Daru, Statsample and Statsample-glm. May -- August 2016.
  \item Mentor for Google Summer of Code 2017. Project: Creating the fastest math libraries for Ruby by using the GPU through OpenCL and ArrayFire. May -- August 2017.
  \item Research assistantship under the supervision of {Dr.\,Oleksandr Gromenko} at the Department of Mathematics, Tulane University, working on novel statistical methods for the analysis of spatio-temporal processes. June -- August 2014.
  \item President of the SIAM student chapter at Tulane University, and organizer of the Graduate Student Colloquium at the Department of Mathematics, Tulane University. September 2014 -- September 2016.
\end{itemize}

%---

\section{PUBLICATIONS}

\bibliographystyle{unsrt} %keep list of references unsorted (i.e. sort manualy in the bib file)
\bibliography{Alexej_Gossmann_CV}
\nocite{*}

%---

\section{CONFERENCE PRESENTATIONS}

\begin{itemize} \itemsep -2pt % Reduce space between items
  \item SlopeCCA and gslopeCCA: sorted L1 penalized canonical correlation analysis (Abstract/Program \#2803W). Presented at the {\it 66th Annual Meeting of The American Society of Human Genetics}, October 19, 2016, Vancouver, Canada.
  \item Identification of Significant Genetic Variants via SLOPE, and its Extension to Group SLOPE (Abstract/Program \#1343F). Presented at the {\it 65th Annual Meeting of The American Society of Human Genetics}, October 9, 2015, Baltimore, MD.
  \item Identification of Significant Genetic Variants via SLOPE, and its Extension to Group SLOPE. The {\it 6th ACM Conference on Bioinformatics, Computational Biology, and Health Informatics}, Atlanta, GA, September 2015.
\end{itemize}

%---

\section{SOFTWARE}

\begin{itemize} \itemsep -2pt % Reduce space between items
    \item \verb!grpSLOPE! -- Group SLOPE (Group Sorted L1 Penalized Estimation) is a penalized linear regression method that is used for adaptive selection of groups of significant predictors in a high-dimensional linear model. The Group SLOPE method can control the (group) false discovery rate at a user-specified level. Project repository: \url{https://github.com/agisga/grpSLOPE}. CRAN: \url{https://cran.r-project.org/web/packages/grpSLOPE/index.html}.
    \item \verb!grpSLOPEMC! -- This is an extension package to the R package grpSLOPE. It contains Monte Carlo based methods for the estimation of the regularizing sequence. Project repository: \url{https://github.com/agisga/grpSLOPEMC}.
    \item \verb!mixed_models! -- Fit statistical linear models with fixed and random effects in Ruby. Project repository: \url{https://github.com/agisga/mixed_models}. RubyGems: \url{https://rubygems.org/gems/mixed_models}.
    \item In my free time I like to contribute to open source software projects. Visit my Github page for the projects that I contribute to: \url{https://github.com/agisga}.
\end{itemize}

%---

\section{EXTRA-CURRICULAR \\ ACTIVITIES}

Elected {\it President} of the SIAM student chapter at Tulane University, September 2014 -- September 2016.
Organized the Graduate Student Colloquium at the Department of Mathematics, Tulane University, September 2014 -- September 2016.

%---

\section{HONORS \& AWARDS}

\begin{itemize}
  \item Reviewers’ Choice Abstract (top 10\% of poster abstracts), 66th Annual Meeting of The American Society of Human Genetics, October 19, 2016, Vancouver, Canada.
  \item SIAM Student Chapter Certificate of Recognition, SIAM (Society for Industrial and Applied Mathematics), April 2015.
\end{itemize}

\vfill
\moveleft.5\hoffset\centerline{Updated: \today}

%----------------------------------------------------------------------------------------

\end{resume}

\end{document}
